% !TEX encoding = UTF-8 Unicode
% !TEX root =  Bachelorarbeit.tex

% Meta-Informationen ------------------------------------------------------------------------------------
%   Definition von globalen Parametern, die im gesamten Dokument verwendet
%   werden können (z.B auf dem Deckblatt etc.).
%
%   ACHTUNG: Wenn die Texte Umlaute oder ein Esszet enthalten, muss der folgende
%            Befehl bereits an dieser Stelle aktiviert werden:
%            \usepackage[latin1]{inputenc}
% -------------------------------------------------------------------------------------------------------
\newcommand{\titel}{How can Wave function Collapse algorithm be used to generate Citys?}
\newcommand{\untertitel}{Considering factors like walkability, population density, and urban functionality}
\newcommand{\untertitelDeckblatt}{Considering factors like walkability, population density, and urban functionality}
\newcommand{\art}{Bachelor-Thesis}
\newcommand{\fachgebiet}{zur Erlangung des akademischen Grades\\ Bachelor of Science (B.\,Sc.) im Studienfach\xspace}
\newcommand{\autor}{Jakob Günster}
\newcommand{\keywords}{Bachelorarbeit, Jakob Günster}
\newcommand{\studienbereich}{Medieninformatik\xspace}
\newcommand{\matrikelnr}{41493}
\newcommand{\erstgutachter}{Prof. Dr. Jens Hahn}
\newcommand{\zweitgutachter}{Florian Rupp}
\newcommand{\jahr}{2024}
\newcommand{\hochschule}{Hochschule der Medien}
\newcommand{\ort}{Stuttgart}
\newcommand{\logo}{hdm-logo.jpg}
\newcommand{\creator}{TeXShop 3.16}
